%% start of file `template.tex'.
%% Copyright 2006-2013 Xavier Danaux (xdanaux@gmail.com).
%
% This work may be distributed and/or modified under the
% conditions of the LaTeX Project Public License version 1.3c,
% available at http://www.latex-project.org/lppl/.


\documentclass[10pt,a4paper,bitstream]{moderncv}

% possible options include font size ('10pt', '11pt' and '12pt'),
% paper size ('a4paper', 'letterpaper', 'a5paper', 'legalpaper',
% 'executivepaper' and 'landscape') and font family ('sans' and
% 'roman')

%\renewcommand{\familydefault}{bitstream-charter} % to set the default font;
%use '\sfdefault' for the default sans serif font, '\rmdefault' for
%the default roman one, or any tex font name

\moderncvstyle{casual}
% style options are 'casual' (default), 'classic', 'oldstyle' and
% 'banking'

\moderncvcolor{deepblue}
% color options 'blue' (default), 'orange', 'green', 'red', 'purple',
% 'grey' and 'black'

\nopagenumbers{}

\usepackage[utf8]{inputenc}

% adjust the page margins
\usepackage[scale=0.85]{geometry}

\name{Sanjoy}{Das}
\title{Software Engineer}
% \address{street and number}{postcode city}{country}
\phone[mobile]{+1-408-816-9041}
\email{sanjoy@playingwithpointers.com}
\homepage{http://playingwithpointers.com}
\github{http://github.com/sanjoy/}
%\photo[64pt][0.6pt]{me}

\newcommand{\tinygap}{\vspace{7 pt}}
\newcommand{\largegap}{\vspace{10 pt}}

\begin{document}

\makecvtitle

\section{Summary}
\tinygap
\cvitem{}{Software engineer particularly interested in the design and
  implementation of compilers and virtual machines.}{}

\largegap
\section{Work Experience}

\tinygap
\cventry{August 2013--Current}{
  Software Engineer, JVM
}{
  Azul Systems
}{
  Sunnyvale, California
}{}{
%
  Compiler engineer on the JVM team.
%
}

\tinygap

\cventry{May 2012--July 2012}{
  Software Engineering Intern
}{
  Google
}{
  München, Germany
}{}{
%
  I interned on the \textbf{V8 / Chrome} team and worked on moving
  some parts of the optimizing Javascript compiler's (Crankshaft's)
  pipeline to a dedicated compiler thread.
%
}

\tinygap
\cventry{July 2011--December 2011}{
  Intern (Off-site)
}{
  Igalia
}{}{}{
%
  I worked on \textbf{gdb}, implementing a new, simpler debug info
  protocol for JIT compilers.
%
}

\tinygap
\cventry{May 2011--September 2011}{
  Google Summer of Code Student
}{
  The LLVM Compiler Infrastructure
}{}{}{
%
  I implemented support for \textbf{segmented stacks} in \textbf{llvm}
  for x86.  This allows llvm generated code to incrementally allocate
  the native thread stack as it runs.
%
}

\largegap
\section{Skills \& Expertise}

\tinygap
\subsection{Skills}
\cvitem{}{%
  {\small My core competencies are virtual machines and compilers.
    Besides that I have dabbled in type-theory and programming
    language semantics.  I'm most comfortable with x86 linux
    systems.}}{}

\tinygap
\subsection{Programming Languages}
\cvitem{}{%
  {\small My love for interesting programming domains trumps my love
    for interesting programming languages; most of the production code
    I've written is in \textbf{C++} and \textbf{C}.  I have used
    \textbf{Haskell} in non-trivial personal projects and am familiar
    with \textbf{Java}, \textbf{Agda} and \textbf{Python}.}}  {}

\largegap
\section{Education}

\tinygap
\vbox{\cventry{2008--2013}{Master of Science}{Indian Institute of Technology Kharagpur}{}{}{
  Mathematics \& Computing}}

\tinygap
\vbox{\cventry{2008--2013}{Bachelor of Science}{Indian Institute of Technology Kharagpur}{}{}{
  Mathematics \& Computing}}

\end{document}
